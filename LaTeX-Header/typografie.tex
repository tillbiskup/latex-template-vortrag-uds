% Inoffizielle LaTeX-Vorlage für Vorträge
% am Lehrstuhl für Physikalischen Chemie und Didaktik der Chemie
% an der Universität des Saarlandes
%
% Diese Vorlage ist lediglich ein Vorschlag, der versucht, sowohl
% typografischen Ansprüchen ansatzweise gerecht zu werden als auch
% nutzbar zu sein.
%
% Jegliche Nutzung auf eigene Verantwortung.
%
% Copyright (c) 2018-19, Till Biskup

%% Anführungszeichen sprachabhängig und "intelligent"
\usepackage[autostyle]{csquotes}

%% typografisch korrekte Tabellen
\usepackage{booktabs}

%% Korrekter Satz von Zahlen und Einheiten
\usepackage{siunitx}

% Komma als Dezimaltrennzeichen im Text (unabhängig von der Eingabe)
\sisetup{output-decimal-marker={,}}

% Befehl für Quellenangaben
\renewcommand*{\thefootnote}{}
\renewcommand*{\footnoterule}{\rule{0cm}{0cm}}
% Quellenangaben frei formatierbar, rechtsbündig unten auf der Seite
\newcommand*{\quelle}[1]{\footnotetext{\vspace*{.5mm}\raggedleft{\tiny #1}}}

%% Pakete für zusätzliche Symbole
\usepackage{pifont}

%% Logische Textauszeichnung
% fremdsprachige Begriffe kursiv
\newcommand*{\foreign}[1]{\emph{#1}}

% LaTeX-Paketnamen in Schreibmaschinenschrift (dicktengleich)
\newcommand*{\package}[1]{\texttt{#1}}

% (LaTeX-)Befehle in Schreibmaschinenschrift (dicktengleich)
\newcommand*{\command}[1]{\texttt{#1}}

% Datei- und Verzeichnisnamen in Schreibmaschinenschrift
\newcommand*{\filename}[1]{\texttt{#1}}
